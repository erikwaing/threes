\documentclass[11pt]{article}
\usepackage{graphicx}    % needed for including graphics e.g. EPS, PS
\topmargin -1.5cm        % read Lamport p.163
\oddsidemargin -0.04cm   % read Lamport p.163
\evensidemargin -0.04cm  % same as oddsidemargin but for left-hand pages
\textwidth 16.59cm
\textheight 21.94cm 
%\pagestyle{empty}       % Uncomment if don't want page numbers
\parskip 7.2pt           % sets spacing between paragraphs
%\renewcommand{\baselinestretch}{1.5} % Uncomment for 1.5 spacing between lines
\usepackage{amsmath}
\usepackage{verbatim}
\usepackage{amsfonts}
\parindent 0pt		 % sets leading space for paragraphs
\author{Erik Waingarten, Fermi Ma, Matt Susskind}
\title{ES.S20 Final Project 1: \\
Strategies for \emph{Threes!}}


\begin{document}         
\maketitle

\section{Game Description}


\emph{Threes!} is a popular puzzle phone game created by Asher Vollmer. The game is a one-player game played on a $4 \times 4$ board of squares, where each non-empty square is labeled with either a 1, 2, or a positive integer of the form $3*2^i$.

The game starts out with 9 of the 16 squares non-empty, where these non-empty squares can take on the values 1, 2, or 3. On each turn, the player chooses to slide the tiles in one of the four directions (up, down, left, or right). This moves each square on the board one unit over in that direction (possibly combining it with another square), unless it is already against the edge of the board or is adjacent to a tile it cannot combine with.

Two square combine by adding their values together if they are a 1 and a 2, or if they both have an equal value of at least 3. If there are more than two numbers in the same row or column that could combine on a single move, the combination occurs between the two squares that are farthest in the direction of the move. For example, if the player slides the squares left, and three 3's are adjacent, the leftmost 3's will combine into a 6. At each move, one new tile (usually a 1, 2, or 3, but sometimes a higher value) slides in.

The game ends when the board is filled with non-empty squares and no further moves can be made. The game is scored when the game ends, where each tile with value $n$ that is not a 1 or 2 contributes $3^{\log_2(n/3)+1}$ to the score and each tile with value $1$ or $2$ contributes $1$ or $2$ respectively. 

New players often end with a score around $1,000$, while experienced players often score between 20,000 and 70,000 points.

\section{Real Gameplay}

In standard \emph{Threes!} gameplay, a well-known strategy is known as the ``corner strategy". The corner strategy essentially tells the player to build up the largest tile so that it stays in one corner of the board, and to never move this tile out of the corner unless the player is stuck otherwise. The general reasoning behind the strategy is that the largest tiles on the board can't combine with other tiles, and so they simply take up space and impede the process of combining squares. Thus, it seems optimal to keep the largest tile in a corner, where it is only adjacent to two other squares (rather than three or four).

We believe that the best algorithm to play \emph{Threes!} won't follow a corner strategy exactly, but the strategy gives us a good idea of what sort of heuristics we should pay attention to.

\section{Analysis}

\subsection{Maximum Score}

The first question we wanted to explore was the idea of a maximum possible score. In theory, one certainly does exist; with an induction-type argument, we can argue that the highest possible tile is $3*2^{15}$. To see why, note that achieving $3*2^{15}$ first requires filling the board of 16 tiles in a way that allows to end up with two $3*2^{14}$ tiles. Consider the second such tile, and note achieving $3*2^{14}$ first requires filling the remaining 15 tiles in a way that leaves us two $3*2^{13}$ tiles. We can repeat this process to show that the maximum score is
\[ 2 + \sum_{i=0}^{14} 3^{i+1} = 2 + (3^{15}-1)/2 = 7174455 \]

We were unable to determine if there actually exists a sequence of input tiles that allows the player to achieve this maximum score. However, it can certainly be done if the board is modified to be a line of 16 tiles and the inputs always come in from the same spot and alternate between 1's and 2's. In real gameplay, the randomness of the input values and input location makes achieving anytiHng close to this upper bound nearly impossible.

% FIX THIS UP
\begin{comment}
The question remains of whether this score is attainable. We note that this is attainable through the following process. Suppose that the tiles entered the board in the following process:\\
\begin{itemize}
\item At the beginning pieces alternate between $1$ and $2$.
\item First, pieces come in from the bottom left, and we push these to the right. 
\begin{itemize}
\item continue this until you are left with $2, 3, 6, 12$ in the bottom row.
\item we can do this with every row (with last value inserted alternating between 1 and 2), and then go down, inserting alternating 1 and 2 at the top in the first column to get 
\[ \begin{array}{cccc}
2 & & & \\
1 & & & \\
2 & & & \\
6 &12&24&48 \end{array} \]
\end{itemize}
\item We now go down until we fill in the whole board like
\[ \begin{array}{cccc} 
2 & & & \\
3 & & & \\
6 & & & \\
12&24 &48&96\end{array} \]
\item We can do a similar process in the other direction to achieve
\[ \begin{array}{cccc}
12&6 & 3& 2 \\
24& & &  \\
48& & & \\
96 & 192 & 284& 568\end{array} \]
\end{itemize}
\end{comment}

\subsection{What Makes a Good Board?}

Another interesting question that arises in \emph{Threes!} is: What does it mean to have a good board? Many times during real gameplay, boards that look like losing boards / bad situations can turn into very good situations in a surprisingly small number of moves. We explore this question of what it means to have a good board by running some experiments. 

We came up with a list of possible heuristics of how to evaluate a board. The plan is to determine which of these heuristics a player should follow to attain a better score. We built an AI for \emph{Threes!} that can play and simulate a player that does a game tree search with pruning, evaluating the board using the different heuristics. 

While the original game of \emph{Threes!} has a degree of uncertainty involved (as the sequence of tiles that will slide in is unknown), we consider a deterministic version of the game. The tiles that enter the board alternate between coming in as $1, 2,$ and $3$ and the tiles come into the first empty spot in the direction that the user moves. Also, in our version of the game, the board begins with all empty tiles.

In this simpler, deterministic variant of the game, the game tree can actually be drawn out beforehand in its entirety (but is still infeasible to actually compute). Each node has a branching factor of at most four, corresponding to the four possible directions tiles can slide at each move.

We considered the following evaluators:

\texttt{MaximizeScore}: Evaluates a board based solely on the score that board would receive, if the game were to end at that point.

\texttt{SumOfSquares}: Evaluates a board based on the sum of the squares of the numbers on the board.

\texttt{SumOfCubes}: Same as \texttt{SumOfSquares}, except with cubes.

\texttt{Gravity}: Evaluates the negative gravitational potential of the board. This biases the board to have the higher numbers on the bottom row.

\texttt{SumOfBottom}: Evaluates a board based on the sum of the tiles on the bottom row.

\texttt{EmptySquares}: Evaluates a board (inversely) based on the number of empty squares.

\texttt{MinOneTwo}: Evaluates a board (inversely) based on the number of 1's and 2's present. 

\texttt{PositionOfHighest}: Evaluates a board based on the position of the highest nonempty tile. This evaluates a board more highly if the highest tile is in a corner or an edge, with the idea that this should favor strategies that push higher tiles to the corners, which is a common strategy used by players. 

\section{Results}

We obtained the following results running the evaluators on the deterministic version of the game with the AI having two look-aheads.

\begin{tabular}{l c c}
\hline\hline % inserting double-line 
Evaluator & \ Score
\\ [0.5ex] 
\hline % inserts single-line 
 
% Entering 1st row 
\texttt{MaximizeScore} & 9135 \\

\texttt{SumOfSquares} &  22977 \\

\texttt{SumOfCubes} &  22977  \\

\texttt{Gravity} & 3837   \\

\texttt{SumOfBottom} & 2818  \\

\texttt{EmptySquares} & 23308  \\

\texttt{MinOneTwo} & 2668\\

\texttt{PositionOfHighest} & 580  \\
 
% [1ex] adds vertical space 
\hline % inserts single-line 
\end{tabular}


It is surprising that maximizing the score did not achieve the highest score and that sum of squares and cubes yielded higher scores. The formula for evaluating the board score always scores the tiles lower than in the sum of squares. This means that the board score underweights the benefit of combining larger values, which seem to be better for survival in this kind of game. 

In the end, maximizing the number of empty squares seems to be the best for achieving high scores in this game. It makes sense that this would be a good strategy. Making empty squares requires combining existing tiles, which creates higher valued tiles. All the other strategies did not do very well. 

\section{Conclusion}

The game of \emph{Threes!} is an interesting game that combines randomization, planning, and strategy. We sought to answer the question of what makes a good board. A series of experiments showed that in a deterministic variant of the game, using two look aheads, the best heuristic to use was to maximize the number of empty squares on the board. It was interesting to see that maximizing the score was not a very good strategy in this game.

It would be interesting to see what would happen in the actual randomized version of the game. We believe that in the actual version of the game, the other strategies that players use become more important and can have a more profound effect on the score.
\end{document}









